% Options for packages loaded elsewhere
\PassOptionsToPackage{unicode}{hyperref}
\PassOptionsToPackage{hyphens}{url}
%
\documentclass[
]{article}
\usepackage{amsmath,amssymb}
\usepackage{iftex}
\ifPDFTeX
  \usepackage[T1]{fontenc}
  \usepackage[utf8]{inputenc}
  \usepackage{textcomp} % provide euro and other symbols
\else % if luatex or xetex
  \usepackage{unicode-math} % this also loads fontspec
  \defaultfontfeatures{Scale=MatchLowercase}
  \defaultfontfeatures[\rmfamily]{Ligatures=TeX,Scale=1}
\fi
\usepackage{lmodern}
\ifPDFTeX\else
  % xetex/luatex font selection
\fi
% Use upquote if available, for straight quotes in verbatim environments
\IfFileExists{upquote.sty}{\usepackage{upquote}}{}
\IfFileExists{microtype.sty}{% use microtype if available
  \usepackage[]{microtype}
  \UseMicrotypeSet[protrusion]{basicmath} % disable protrusion for tt fonts
}{}
\makeatletter
\@ifundefined{KOMAClassName}{% if non-KOMA class
  \IfFileExists{parskip.sty}{%
    \usepackage{parskip}
  }{% else
    \setlength{\parindent}{0pt}
    \setlength{\parskip}{6pt plus 2pt minus 1pt}}
}{% if KOMA class
  \KOMAoptions{parskip=half}}
\makeatother
\usepackage{xcolor}
\usepackage[margin=1in]{geometry}
\usepackage{color}
\usepackage{fancyvrb}
\newcommand{\VerbBar}{|}
\newcommand{\VERB}{\Verb[commandchars=\\\{\}]}
\DefineVerbatimEnvironment{Highlighting}{Verbatim}{commandchars=\\\{\}}
% Add ',fontsize=\small' for more characters per line
\usepackage{framed}
\definecolor{shadecolor}{RGB}{248,248,248}
\newenvironment{Shaded}{\begin{snugshade}}{\end{snugshade}}
\newcommand{\AlertTok}[1]{\textcolor[rgb]{0.94,0.16,0.16}{#1}}
\newcommand{\AnnotationTok}[1]{\textcolor[rgb]{0.56,0.35,0.01}{\textbf{\textit{#1}}}}
\newcommand{\AttributeTok}[1]{\textcolor[rgb]{0.13,0.29,0.53}{#1}}
\newcommand{\BaseNTok}[1]{\textcolor[rgb]{0.00,0.00,0.81}{#1}}
\newcommand{\BuiltInTok}[1]{#1}
\newcommand{\CharTok}[1]{\textcolor[rgb]{0.31,0.60,0.02}{#1}}
\newcommand{\CommentTok}[1]{\textcolor[rgb]{0.56,0.35,0.01}{\textit{#1}}}
\newcommand{\CommentVarTok}[1]{\textcolor[rgb]{0.56,0.35,0.01}{\textbf{\textit{#1}}}}
\newcommand{\ConstantTok}[1]{\textcolor[rgb]{0.56,0.35,0.01}{#1}}
\newcommand{\ControlFlowTok}[1]{\textcolor[rgb]{0.13,0.29,0.53}{\textbf{#1}}}
\newcommand{\DataTypeTok}[1]{\textcolor[rgb]{0.13,0.29,0.53}{#1}}
\newcommand{\DecValTok}[1]{\textcolor[rgb]{0.00,0.00,0.81}{#1}}
\newcommand{\DocumentationTok}[1]{\textcolor[rgb]{0.56,0.35,0.01}{\textbf{\textit{#1}}}}
\newcommand{\ErrorTok}[1]{\textcolor[rgb]{0.64,0.00,0.00}{\textbf{#1}}}
\newcommand{\ExtensionTok}[1]{#1}
\newcommand{\FloatTok}[1]{\textcolor[rgb]{0.00,0.00,0.81}{#1}}
\newcommand{\FunctionTok}[1]{\textcolor[rgb]{0.13,0.29,0.53}{\textbf{#1}}}
\newcommand{\ImportTok}[1]{#1}
\newcommand{\InformationTok}[1]{\textcolor[rgb]{0.56,0.35,0.01}{\textbf{\textit{#1}}}}
\newcommand{\KeywordTok}[1]{\textcolor[rgb]{0.13,0.29,0.53}{\textbf{#1}}}
\newcommand{\NormalTok}[1]{#1}
\newcommand{\OperatorTok}[1]{\textcolor[rgb]{0.81,0.36,0.00}{\textbf{#1}}}
\newcommand{\OtherTok}[1]{\textcolor[rgb]{0.56,0.35,0.01}{#1}}
\newcommand{\PreprocessorTok}[1]{\textcolor[rgb]{0.56,0.35,0.01}{\textit{#1}}}
\newcommand{\RegionMarkerTok}[1]{#1}
\newcommand{\SpecialCharTok}[1]{\textcolor[rgb]{0.81,0.36,0.00}{\textbf{#1}}}
\newcommand{\SpecialStringTok}[1]{\textcolor[rgb]{0.31,0.60,0.02}{#1}}
\newcommand{\StringTok}[1]{\textcolor[rgb]{0.31,0.60,0.02}{#1}}
\newcommand{\VariableTok}[1]{\textcolor[rgb]{0.00,0.00,0.00}{#1}}
\newcommand{\VerbatimStringTok}[1]{\textcolor[rgb]{0.31,0.60,0.02}{#1}}
\newcommand{\WarningTok}[1]{\textcolor[rgb]{0.56,0.35,0.01}{\textbf{\textit{#1}}}}
\usepackage{graphicx}
\makeatletter
\def\maxwidth{\ifdim\Gin@nat@width>\linewidth\linewidth\else\Gin@nat@width\fi}
\def\maxheight{\ifdim\Gin@nat@height>\textheight\textheight\else\Gin@nat@height\fi}
\makeatother
% Scale images if necessary, so that they will not overflow the page
% margins by default, and it is still possible to overwrite the defaults
% using explicit options in \includegraphics[width, height, ...]{}
\setkeys{Gin}{width=\maxwidth,height=\maxheight,keepaspectratio}
% Set default figure placement to htbp
\makeatletter
\def\fps@figure{htbp}
\makeatother
\setlength{\emergencystretch}{3em} % prevent overfull lines
\providecommand{\tightlist}{%
  \setlength{\itemsep}{0pt}\setlength{\parskip}{0pt}}
\setcounter{secnumdepth}{-\maxdimen} % remove section numbering
\ifLuaTeX
  \usepackage{selnolig}  % disable illegal ligatures
\fi
\IfFileExists{bookmark.sty}{\usepackage{bookmark}}{\usepackage{hyperref}}
\IfFileExists{xurl.sty}{\usepackage{xurl}}{} % add URL line breaks if available
\urlstyle{same}
\hypersetup{
  pdftitle={Proyecto},
  pdfauthor={Angélica Villagómez},
  hidelinks,
  pdfcreator={LaTeX via pandoc}}

\title{Proyecto}
\author{Angélica Villagómez}
\date{2024-04-02}

\begin{document}
\maketitle

\hypertarget{fuente-de-datos}{%
\subsection{Fuente de datos}\label{fuente-de-datos}}

Fuente:
\href{https://www.siicyt.gob.mx/index.php/estadisticas/bases-de-datos-abiertas?id=96}{Conacyt}

\begin{Shaded}
\begin{Highlighting}[]
\CommentTok{\# carga de Datos}

\NormalTok{datos }\OtherTok{\textless{}{-}} \FunctionTok{read.csv}\NormalTok{(}\StringTok{"Becas\_Nuevas\_2014.csv"}\NormalTok{)}

\CommentTok{\#datos \textless{}{-} read.csv("C:\textbackslash{}Users\textbackslash{}Angie\textbackslash{}Desktop\textbackslash{}Proyecto\_R\_AMVC\textbackslash{}Becas\_Nuevas\_2014.csv",header = TRUE, sep = ",", stringsAsFactors = FALSE)}
\FunctionTok{library}\NormalTok{(ggplot2)}
\end{Highlighting}
\end{Shaded}

\begin{verbatim}
## Warning: package 'ggplot2' was built under R version 4.3.3
\end{verbatim}

\begin{Shaded}
\begin{Highlighting}[]
\FunctionTok{library}\NormalTok{(modeest)}
\end{Highlighting}
\end{Shaded}

\begin{verbatim}
## Warning: package 'modeest' was built under R version 4.3.3
\end{verbatim}

\hypertarget{objetivo}{%
\section{1 Objetivo}\label{objetivo}}

Aplicar los conceptos aprendidos durante el curso de R, donde utilizare
datos obtnidos de Conacyt los cuales son abiertos para su consulta, este
archivo contiene información de las becas que ha ofrecido a diversas
instituciones durante el 2014.

\hypertarget{justificaciuxf3n.}{%
\section{2 Justificación.}\label{justificaciuxf3n.}}

Analizarla base de datos, donde permitirá darnos información de a que
instituciones se les brinda mas apoyo de becas.

\hypertarget{analisis-xploratorio-de-los-datos}{%
\section{3 Analisis xploratorio de los
datos}\label{analisis-xploratorio-de-los-datos}}

A continuación se mostrará la media de las instituciones a las cuales
Conacyt les ha brindado algún tipo de beca.

\begin{Shaded}
\begin{Highlighting}[]
\FunctionTok{head}\NormalTok{(datos)}
\end{Highlighting}
\end{Shaded}

\begin{verbatim}
##    APELLIDO.PATERNO APELLIDO.MATERNO        NOMBRE Num_area
## 1 FERNANDEZ DE LARA           TORRES  PAOLA AYERIM        3
## 2             CASAS          ACEVEDO         AARON        7
## 3           ANTONIO             RUIZ          ABIF        7
## 4          MONTALVO         MARTINEZ       ABIGAIL        1
## 5             COLIN          AGUILAR ABILENE GISEH        2
## 6            GARCIA          ALVAREZ ABISH MARIANA        2
##                       AREA.DE.CONOCIMIENTO NIVEL Num_institucion
## 1          III. MEDICINA Y CS. DE LA SALUD     3               3
## 2                         VII. INGENIERIAS     2             101
## 3                         VII. INGENIERIAS     2             149
## 4 I. FISICO MATEMATICAS Y CS. DE LA TIERRA     2              79
## 5                   II. BIOLOGIA Y QUIMICA     2               1
## 6                   II. BIOLOGIA Y QUIMICA     2              52
##                          INSTITUCION.DESTINO  ENTIDAD.FEDERATIVA Num_entidad
## 1                                       BUAP              PUEBLA          21
## 2                                 IT DURANGO             DURANGO          10
## 3               SISTEMAS TECNICOS DE CONTROL              PUEBLA          21
## 4                                     INIFAP             JALISCO          14
## 5 AGENCIA ALEMANA DE COOPERACION TECNICA-GIZ BAJA CALIFORNIA SUR           3
## 6                                     ECOSUR             CHIAPAS           7
##   GENERO
## 1      1
## 2      2
## 3      2
## 4      1
## 5      1
## 6      1
\end{verbatim}

\begin{Shaded}
\begin{Highlighting}[]
\CommentTok{\#media}
\CommentTok{\# Con la media obtenemos el promedio del numero de instituciones a las cuales les ha brindado una beca.}
\FunctionTok{mean}\NormalTok{(datos}\SpecialCharTok{$}\NormalTok{Num\_institucion)}
\end{Highlighting}
\end{Shaded}

\begin{verbatim}
## [1] 124.2837
\end{verbatim}

\begin{Shaded}
\begin{Highlighting}[]
\CommentTok{\#mediana}
\CommentTok{\# Con la mediana obtenemos el valor que se encuentra justo en medio cuando se ordenan los datos, es decir cuando se ordenan las instituciones.}

\FunctionTok{median}\NormalTok{(datos}\SpecialCharTok{$}\NormalTok{Num\_institucion)}
\end{Highlighting}
\end{Shaded}

\begin{verbatim}
## [1] 155.5
\end{verbatim}

\hypertarget{estadistica-inferencial}{%
\section{4 Estadistica inferencial}\label{estadistica-inferencial}}

\begin{Shaded}
\begin{Highlighting}[]
\CommentTok{\# Becas brindadas a la UNAM}



\CommentTok{\#varianza}
\FunctionTok{var}\NormalTok{(datos}\SpecialCharTok{$}\NormalTok{Num\_entidad)}
\end{Highlighting}
\end{Shaded}

\begin{verbatim}
## [1] 63.19078
\end{verbatim}

\hypertarget{resultados}{%
\section{5 Resultados}\label{resultados}}

\begin{Shaded}
\begin{Highlighting}[]
\CommentTok{\#suma de total por género}
\CommentTok{\# class(datos$GENERO)}

\CommentTok{\#Numero de datos}
\NormalTok{masculino }\OtherTok{\textless{}{-}} \FunctionTok{sum}\NormalTok{(datos}\SpecialCharTok{$}\NormalTok{GENERO}\SpecialCharTok{==}\DecValTok{2}\NormalTok{)}
\NormalTok{masculino}
\end{Highlighting}
\end{Shaded}

\begin{verbatim}
## [1] 602
\end{verbatim}

\begin{Shaded}
\begin{Highlighting}[]
\NormalTok{femenino }\OtherTok{\textless{}{-}} \FunctionTok{sum}\NormalTok{(datos}\SpecialCharTok{$}\NormalTok{GENERO}\SpecialCharTok{==}\DecValTok{1}\NormalTok{)}
\NormalTok{femenino}
\end{Highlighting}
\end{Shaded}

\begin{verbatim}
## [1] 674
\end{verbatim}

\begin{Shaded}
\begin{Highlighting}[]
\NormalTok{sexo }\OtherTok{\textless{}{-}} \FunctionTok{c}\NormalTok{(masculino,femenino)}
\FunctionTok{pie}\NormalTok{(sexo)}
\end{Highlighting}
\end{Shaded}

\includegraphics{Proyecto_AMVC_files/figure-latex/unnamed-chunk-4-1.pdf}

\begin{Shaded}
\begin{Highlighting}[]
\NormalTok{FISICOMATEMATICAS}\OtherTok{\textless{}{-}} \FunctionTok{sum}\NormalTok{(datos}\SpecialCharTok{$}\NormalTok{Num\_area}\SpecialCharTok{==}\DecValTok{1}\NormalTok{)}
\NormalTok{BIOLOGIAQUIMICA}\OtherTok{\textless{}{-}} \FunctionTok{sum}\NormalTok{(datos}\SpecialCharTok{$}\NormalTok{Num\_area}\SpecialCharTok{==}\DecValTok{2}\NormalTok{)}
\NormalTok{MEDICINADELASALUD}\OtherTok{\textless{}{-}} \FunctionTok{sum}\NormalTok{(datos}\SpecialCharTok{$}\NormalTok{Num\_area}\SpecialCharTok{==}\DecValTok{3}\NormalTok{)}
\NormalTok{HUMANIDADESDECONDUCTA}\OtherTok{\textless{}{-}} \FunctionTok{sum}\NormalTok{(datos}\SpecialCharTok{$}\NormalTok{Num\_area}\SpecialCharTok{==}\DecValTok{4}\NormalTok{)}
\NormalTok{CIENCIASSOCIALES}\OtherTok{\textless{}{-}} \FunctionTok{sum}\NormalTok{(datos}\SpecialCharTok{$}\NormalTok{Num\_area}\SpecialCharTok{==}\DecValTok{5}\NormalTok{)}
\NormalTok{BIOTECNOLOGIAAGROPECUARIAS}\OtherTok{\textless{}{-}} \FunctionTok{sum}\NormalTok{(datos}\SpecialCharTok{$}\NormalTok{Num\_area}\SpecialCharTok{==}\DecValTok{6}\NormalTok{)}
\NormalTok{INGENIERIAS}\OtherTok{\textless{}{-}} \FunctionTok{sum}\NormalTok{(datos}\SpecialCharTok{$}\NormalTok{Num\_area}\SpecialCharTok{==}\DecValTok{7}\NormalTok{)}

\NormalTok{Carreras}\OtherTok{\textless{}{-}}\FunctionTok{c}\NormalTok{(FISICOMATEMATICAS,}
\NormalTok{            BIOLOGIAQUIMICA,}
\NormalTok{            MEDICINADELASALUD,}
\NormalTok{            HUMANIDADESDECONDUCTA,}
\NormalTok{            CIENCIASSOCIALES,}
\NormalTok{            BIOTECNOLOGIAAGROPECUARIAS,}
\NormalTok{            INGENIERIAS)}
\FunctionTok{barplot}\NormalTok{(Carreras,}\AttributeTok{col=}\FunctionTok{rainbow}\NormalTok{(}\DecValTok{7}\NormalTok{),}\AttributeTok{main=}\StringTok{"Áreas de estudio"}\NormalTok{, }\AttributeTok{xlab =} \StringTok{"Área de conocimiento"}\NormalTok{, }\AttributeTok{ylab =}\StringTok{"Número de becas"}\NormalTok{)}
\end{Highlighting}
\end{Shaded}

\includegraphics{Proyecto_AMVC_files/figure-latex/unnamed-chunk-5-1.pdf}

\begin{Shaded}
\begin{Highlighting}[]
\NormalTok{UNAM}\OtherTok{\textless{}{-}}\FunctionTok{sum}\NormalTok{(datos}\SpecialCharTok{$}\NormalTok{INSTITUCION.DESTINO}\SpecialCharTok{==}\StringTok{"UNAM"}\NormalTok{)}
\NormalTok{UNAM}
\end{Highlighting}
\end{Shaded}

\begin{verbatim}
## [1] 175
\end{verbatim}

\begin{Shaded}
\begin{Highlighting}[]
\NormalTok{UAM}\OtherTok{\textless{}{-}}\FunctionTok{sum}\NormalTok{(datos}\SpecialCharTok{$}\NormalTok{INSTITUCION.DESTINO}\SpecialCharTok{==}\StringTok{"UAM"}\NormalTok{)}
\NormalTok{UAM}
\end{Highlighting}
\end{Shaded}

\begin{verbatim}
## [1] 44
\end{verbatim}

\begin{Shaded}
\begin{Highlighting}[]
\NormalTok{IPN}\OtherTok{\textless{}{-}}\FunctionTok{sum}\NormalTok{(datos}\SpecialCharTok{$}\NormalTok{INSTITUCION.DESTINO}\SpecialCharTok{==}\StringTok{"IPN"}\NormalTok{)}
\NormalTok{IPN}
\end{Highlighting}
\end{Shaded}

\begin{verbatim}
## [1] 66
\end{verbatim}

\begin{Shaded}
\begin{Highlighting}[]
\NormalTok{total }\OtherTok{\textless{}{-}} \FunctionTok{nrow}\NormalTok{(datos)}
\NormalTok{total}
\end{Highlighting}
\end{Shaded}

\begin{verbatim}
## [1] 1276
\end{verbatim}

\begin{Shaded}
\begin{Highlighting}[]
\NormalTok{OTROS }\OtherTok{\textless{}{-}}\NormalTok{total}\SpecialCharTok{{-}}\NormalTok{(UNAM}\SpecialCharTok{+}\NormalTok{UAM}\SpecialCharTok{+}\NormalTok{IPN)}
\NormalTok{OTROS}
\end{Highlighting}
\end{Shaded}

\begin{verbatim}
## [1] 991
\end{verbatim}

\begin{Shaded}
\begin{Highlighting}[]
\NormalTok{Instituciones}\OtherTok{\textless{}{-}}\FunctionTok{c}\NormalTok{(UNAM,UAM,IPN,OTROS)}
\FunctionTok{barplot}\NormalTok{(Instituciones,}\AttributeTok{col=}\FunctionTok{rainbow}\NormalTok{(}\DecValTok{4}\NormalTok{),}\AttributeTok{main=}\StringTok{"Instituciones"}\NormalTok{, }\AttributeTok{xlab =} \StringTok{"Instituciones"}\NormalTok{, }\AttributeTok{ylab =}\StringTok{"Número de becas"}\NormalTok{)}
\end{Highlighting}
\end{Shaded}

\includegraphics{Proyecto_AMVC_files/figure-latex/unnamed-chunk-6-1.pdf}

\end{document}
